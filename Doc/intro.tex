\chapter{Introduction}
\setcounter{page}{1}
\pagenumbering{arabic}

% ------------------------------------------
\section{Purpose and goals of {\gs}}

{\gs} is an interactive editor for musical scores.  It is interactive
in the sense that the entire score is reformatted after each keystroke
on the part of the user.  There is no special button to click or
command to type in order for the final layout to appear; the final
layout is always the one that is presented. 

One of the main goal of {\gs} is to provide a program that can produce
high-quality scores.  For that reason, we have put in some effort to
respect at least one set of rules of musical typography (or
\emph{engraving}\index{engraving}).  But music engraving is hard and
littered with very complex rules that do not necessarily cover all
cases.  For that reason, {\gs} also provides \unimp{We ultimately
  intend to provide a large spectrum of ways for the user to tweak the
  layout decisions, but for now, only automatic layout exists.} a
number of ways in which the user can help the layout engine make the
right decision. 

We wanted {\gs} to be extensible at a number of level.  The ordinary
end-user should be able to add and redefine keystrokes, to create
macros, and to alter global parameters that affect the end result.
The more advanced end-user should be able to write more complex
extensions.  Usually, programs that permit this use a {\gs} scripting
language, which is usually different from the programming language
that was used to implement the basic functionality of the program.
{\gs} has a different approach.  The entire program is written in
{\commonlisp} which both is as fast as a lower-level language and provides the
ability to dynamically replace any part of the program.  

Since {\gs} makes extensive use of graphics and related functionality,
we needed a good library for providing such functionality.  While the
{\commonlisp} standard does not talk about such libraries, there as been a
semi-standard library available for many years and which only recently
exists in a freely distributable version, namely {\clim}\index{\clim},
which stands for the Common Lisp Interface Manager.  Many of the basic
functions of {\gs} are due to {\clim}, and the advances {\gs} user can
take advantage of this fact in order to improve the program. 

We deliberately did not try commercial score editors so as not to be
influenced by their view of the editing task.  This fact has both
positive and negative consequences.  On the positive side, {\gs} (we
hope) provides novel ideas concerning user interaction.  We are also
hoping to avoid being sued by having a program that looks like a copy
of some commercial one.  On the negative side, users familiar with
existing commercial editors will not be able to take advantage of this
knowledge when using {\gs}.  Also, some commercial editors probably
have some very good ideas that would have been interesting to take
advantage of in {\gs}, which because of this policy will not be
possible. 

{\gs} is intended to be ``user friendly'', but since people do not
agree on a definition of the term, this is what we mean here (largely
inspired by Cooper).  We do not try to cater to new or casual users.
Instead we try to make the average user with at least a few days of
experience as efficient as possible.  This might mean that the user
has to occasionally read the manual in order to figure out how to
operate {\gs}.  We think that is a reasonable price to pay in order to
be efficient. 
% ------------------------------------------
\section{{\gs} is free software}

By ``free software'', we mean that {\gs} can be used, redistributed,
and improved by anyone.  Technically, the main parts of {\gs} are
distributed according to the GNU General Public License (GPL).
Without going into detail here, this means that improvements to the
software must (if distributed at all) also be distributed according to
the GPL.  Ordinary users are not affected by the license, which covers
only improvements and redistribution.  

Some parts of {\gs} are distributed according to the GNU Lesser
General Public License (LGPL).  Again, without going into detail, this
license is used for more general-purpose code such as extensions to
{\clim}, and allows a user to distribute it as part of commercial
software.  As with the GPL, contributions to this code must be
redistributed according to the same license. 

%% % ------------------------------------------
%% \section{Public}

% ------------------------------------------
\section{Conventions}

This manual mostly reflects the current state of things.  Occasionally,
we documented a feature that is not yet implemented, so that you would
have an idea what it would be like.  In that case, we added a footnote
to indicate that the feature is not yet implemented. 

% ------------------------------------------
\section{Releases and release numbering}

Releases are numbered A.B, where A is the major release number and B
is the minor release number or version number.  The first release is
0.1.  We will increase the minor release number with at least one for
each new version of the program made available to the public.  The
major release number is incremented only when a major overhaul has
been done, or when large chunks of code have been added.  

Major release 0 is not intended for prime time, but only for end users
who want to get an idea of the direction we are heading.  Release 1.0
will be declared when the program is usable and stable enough for
general use. 

There is no upper bound on the minor release number.

% ------------------------------------------
\section{Contributions}

We accept contributions in the form of bug fixes, improvements to the
existing code, and new code.  Since {\gs} is distributed according to
the GPL, nothing prevents any user to add new code and distribute the
result.  We hope that this will not happen immediately, since we would
like to guarantee some consistency of the program, both regarding its
external appearance and its internal organization.  For that reason,
it would be preferable to discuss with us before contributing some
code. 

Contributors must have a reasonable level of knowledge of {\commonlisp}.  As
with any language, it is possible to write poor code with {\commonlisp}.  It
is our goal to try to keep the code at a reasonably good level.  For
that reason, contributors are encouraged to put some effort into the
quality of the code, to make sure it compiles without warnings (as
much as possible) and to make sure it works.

We also accept contributions to this document.  We intend to use some
kind of free license for this document, probably the GNU Free
Documentation License, but we have not had time to figure out how it
works yet. 
% ------------------------------------------
\section{Organization of this document}

This document is organized in several \emph{parts}.  

The \emph{getting started} part is intended for users who look at
{\gs} for the first time, and who want to know how to use the most
fundamental aspects of it, in particular the basic architecture of the
program.  It is not intended to be a complete user's guide, but
instead structured around a few typical use cases encountered by
users. 

The \emph{reference manual} is the part that is most useful to the
user that has graduated beyond the first impression, and who want to
know more about the features of {\gs}.  The structure of this part is
not meant to be pedagogical, but instead a complete enumeration of the
features of {\gs} that a moderately experienced users needs to know. 

The \emph{internals} part is intended for advanced users who need to
know how to customize {\gs} in non-trivial ways, as well as for the
programmer who would like to extend or modify {\gs}.

Finally, the \emph{appendices} part provides information on topics
that are neither related to the user of {\gs}, nor to the programmer
who wants to extend it, but to readers who simply want to know more
about music engraving or about how {\gs} was created, and why it was
created that way. 

