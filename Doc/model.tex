\chapter{The user model of {\gs}}
\label{user-model}

In order to be an effective user of {\gs}, you need to understand its
\emph{user model} \index{user model}, i.e., the way the user is
supposed to think about the structure of the score being created.

\section{Buffer}
\label{model-buffer}

The basic abstraction of {\gs} is that of a \emph{buffer}
\index{buffer}.  The buffer is the in-memory representation of an
entire musical score.  {\gs} can have several buffers simultaneously
in memory.\unimp{It is not hard to implement, but it just has not
  been done yet --RS 2003-08-16}.  

Loading a {\gs} file into the editor creates a new buffer, and a
buffer can be saved either to the file from which it was initially
loaded, or to some other file.  A new buffer can be created (which
would then not yet be associated with a file).  

The page layout algorithm of {\gs} has been designed to be able to
cope with very large buffers, while still updating the display in real
time.  Several hundreds of pages should not be a problem.  Though, if
you change some global parameter which effects the complete layout of
the entire score, such as the spacing style, and you have a very large
buffer, you might have to wait a while before {\gs} has completed the
calculation of the new page layout.  Ordinary local changes that do
not impact page layout elsewhere only require a very small amount of
work for {\gs}, so updating the screen should be immediate. 

\section{Segment}
\label{model-segment}

A buffer is divided into smaller pieces known as \emph{segments}
\index{segment}.  A segment roughly corresponds to a musical phrase.
Segments do not overlap, or only slightly.  {\gs} contains a number of
useful commands that operate on an entire segment.  You can copy a
segment or listen to it played as a {\midifile}.

\section{Layer}

A segment can have several \emph{layers}\index{layer}.  The layers of
a segment are temporally superimposed.  You might think of a layer as
corresponding roughly to a \emph{voice} \index{voice} or a \emph{part}
\index{part} of the music.  

The difference between a layer and a part ({\gs} does not know about
parts) is that a layer has a particular \emph{instrument}
\index{instrument} assigned to it, whereas a part can have several
different instruments, for instance when the same musician plays
several instruments (though not simultaneously).  This different poses
no problem, because it is possible to organize {\gs} layers so that
their musical material does not overlap in time, even though layers
conceptually do. 

\section{Slice}

Each layer has exactly three parts, referred to as \emph{slices}
\index{slice}, the \emph{head slice}\index{head slice}
\index{slice!head|)}, the \emph{body slice}\index{body slice}
\index{slice!body|)}, and the \emph{tail slice}\index{tail slice}
\index{slice!tail|)}. \unimp{We only use the body slice at the moment,
  which of course is the most important one.  However, the head slice
  is important for [and I do not know what this is called in English]
  the few notes that might logically belong to a phrase, but that
  temporally precedes it.  I have no idea, though, what the tail slice
  might be useful for. --RS 2003-08-16}

\section{Element}

Each slice is a sequence of \emph{elements}\index{element}.  There are
different kinds of element, \emph{clusters}\index{cluster},
\emph{silences}\index{silence}, and \emph{measure bars}\index{measure
bar}.  For the remainder of this section, we ignore the possibility of
an element being a measure bar, since a measure bar does not have any
interesting properties such as duration.

In the body slice, the first element starts at time zero from the
beginning of the slice, and each subsequent element starts when the
preceding one ends.  The same thing is true for the tail
slice. \unimp{We do not yet have a tail slice} The head slice is
different, in that it is aligned from the \emph{end of the preceding
slice}.  \unimp{We do not yet have a head slice}.

Each element (except the measure bar) has a particular \emph{duration}
\index{duration}.  The duration is indicated by a \emph{notehead}
\index{notehead} and a \emph{number of dots}. \index{number of dots}
\index{dots!number of|)} Notice that the notehead property exists even
if the element is a silence.  It is used to indicate what kind of
character to use for the silence.


\section{Cluster}

A \emph{cluster}\index{cluster} is roughly equivalent to a chord, in
that it contains all the notes that should be played simultaneously
within a layer.  In addition to the properties inherited from the
element, each cluster has a \emph{stem direction} \index{stem
direction} \index{direction!of stem|)}.

Clusters have zero or more \emph{notes}.  When there are zero notes in
the cluster, it is not displayed at all, but the layout algorithm
takes it into account with its ordinary duration.  This allows the
user to introduce spaces in the music that are not filled with
explicit silence. 

The stem direction indicated in the cluster may or may not be
respected by the layout algorithm.  First, if the stem direction is
\emph{auto}, the layout algorithm will use its knowledge about music
engraving to display the cluster either with the stem up or with the
stem down.  When the stem direction is \emph{down} or \emph{up} as
chosen by the user, the layout algorithm always respects this choice
when the cluster is not part of a beam group.  Currently, all the
stems of a beam group are displayed in the same direction (we will
change this in the future, at least for beam groups where the elements
are on different staves).  Thus, when the cluster is part of a beam
group the stem directions used are determined by the first element of
the group.  When the first element of the group has stem direction
\emph{auto}, then {\gs} will compute the best stem direction for all
the elements of the group.  When the stem direction of the first
element is either \emph{down} or \emph{up}, {\gs} will honor that
choice.  Obviously, in this case, stem directions of the other
elements are ignored. 

\section{Silence}

A silence is an other example of an element.  Although always
displayed with the traditional characters, a silence still has a
notehead, which is used in this case to compute the duration of the
silence, and thus what character to use to display it and to determine
spacing. 

\section{Measure bar}

A measure bar is another kind of element.  It has no interesting
properties in itself.  Apart from being displayed, it is used to align
musical material in different layers.  Thus, {\gs} does not bother to
verify that each layer has the same duration between two measure
bars.  It simply lines up every measure bar in every layer.  If some
layer has a measure that has a shorter duration than the corresponding
measure in a different layer, the shorter one will be padded by empty
space so that they line up. 

Also, {\gs} does not verify that there is the same number of measure
bars in each layer.  When a layer runs out of measure bars, the
remaining ones are lined up without taking into account the layer that
ran out. 
