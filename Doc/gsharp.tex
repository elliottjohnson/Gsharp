\documentclass[11pt]{book}
\newcommand{\Comment}[1]{\begin{center}\tt #1 \end{center}}
% \usepackage{doublespace}
\usepackage[T1]{fontenc}
\usepackage[latin1]{inputenc}
\usepackage{alltt}
\usepackage{moreverb}
\usepackage{epsfig}
\usepackage{makeidx}

\setlength{\parskip}{0.3cm}
\setlength{\parindent}{0cm}

\def\lispout#1{\underline{#1}}
\def\commonlisp{Common Lisp}
\def\scheme{Scheme}
\def\gs{Gsharp}
\def\midifile{Midifile}
\def\midi{MIDI}
\def\obseq{Obseq}
\def\clos{CLOS}
\def\clim{CLIM}
\def\rs{Robert Strandh}
\def\kerhoas{Laurent Kerhoas}
\def\rms{Richard Stallman}
\def\lisp{Lisp}
\def\tcl{Tcl}
\def\elk{Elk}
\def\gnu{GNU}
\def\guile{Guile}
\def\scm{SCM}
\def\xwin{the X Window System}
\def\freetype{FreeType}
\def\metafont{METAFONT}
\def\bs{$\backslash$}
\def\lispobj#1{\textsl{#1}}
\def\lispobjindex#1{\lispobj{#1}\index{#1@\lispobj{#1}}}
\def\syntax#1{\texttt{#1}}
\def\metavar#1{\textit{#1}}
\def\unimp#1{\footnote{This feature is not yet implemented.  #1}}

\def\kbd#1{`#1'}
\def\command#1{``#1''}

\def\inputfig#1{\input #1}
\def\inputtex#1{\input #1}

\newenvironment{itemize0}{
\begin{itemize}
\setlength{\parskip}{0cm}%
}
{\end{itemize}}

\newenvironment{enumerate0}{
\begin{enumerate}
\setlength{\parskip}{0cm}%
}
{\end{enumerate}}

\newenvironment{smalltt}{
\begin{alltt}
\small
}
{\end{alltt}}

\inputtex{spec-macros.tex}

\title{{\Huge {\gs}}\\{\huge The Extensible, Interactive Score Editor}}

\author{{\Large Robert Strandh}}

\date{September 2003}

\makeindex
\begin{document}
\pagenumbering{roman}

\maketitle

\newpage

\vspace*{\fill}
\copyright{} 2003, Robert Strandh

I intend to distribute this document according to some kind of free
license, such as the GNU Free Documentation License, but I have not
had the time to figure out how it works yet. 

\newpage

{\setlength{\parskip}{0cm}
\tableofcontents}

\inputtex{intro.tex}

% ***********************************************************
\part{Getting started}

%======================================================
\chapter{Installing and running {\gs}}

\section{Installing {\gs}}

We have not worked on the installation procedure at all, and in fact,
we have not decided exactly in what form the releases will be made
available to the public.  For that reason, these instructions may
change in the future. 

{\gs} uses a number of programs and libraries that you need to install
before installing {\gs} itself.


Most importantly, you will need an implementation of {\commonlisp}.
We recommend SBCL on GNU/Linux x86 and OpenMCL on MacOS PPC.

To install the libraries and {\gs} itself, you will need ASDF, which
comes with some {\commonlisp} implementations.  If not, you will have
to install it first. 

You will need a copy of the Flexichain library which you can find on
common-lisp.net.  

You will need to download a copy of McCLIM, the free implementation of
the Common Lisp Interface Manager.  Again, you will find it on
common-lisp.net.

{\gs} uses a library on top of McCLIM called ESA (for Emacs-Style
Application).  For now, the ESA library is distributed with {\gs}, but
that might change in the future, in which case you will need a copy of
it as well, again from common-lisp.net.

If you have downloaded a \emph{tar}-file of {\gs}, you need to untar
it (using something like \texttt{tar xvf gsharp.tar}) in some
directory.  If you have downloaded {\gs} from a CVS repository, you
already have the directory structure required.

First make the fonts.  In the \texttt{Fonts} subdirectory, type
\texttt{make}. 

To compile and load {\gs}, you have two possibilities (as with most of
the libraries): either load the \texttt{gsharp.asd} file manually, or
put the directory in which {\gs} resides in the
\texttt{asdf:*central-registry*} list of software that can compiled
and installed using ASDF.

Next, follow the instructions for installing ASDF, Flexichain, and
McCLIM.  When you have all these libraries in a running {\lisp} image,
type: 

\texttt{(asdf:operate 'asdf:load-op :gsharp)}

which will compile and load all the files of {\gs}. To start {\gs} type 

\texttt{(gsharp::gsharp)}

\section{The different editing panes}

When you start {\gs}, you should see a number of windows (called
\emph{panes}) that are described briefly in this section.

\subsection{The score pane}
\label{sec-score-pane}

The largest pane is the \emph{score pane}\index{score
  pane}\index{pane!score|)} where you will do most of the editing,

When prompted for some item on display in the score pane, you can
usually click on it.  For instance, if you have issued a command that
operates on a particular staff, you can usually click on a visual
representation of the staff in the score pane to satisfy the request
for the staff. 

\subsection{The minibuffer}
\label{sec-minibuffer}

The pane at the bottom is called the
\emph{minibuffer}\index{minibuffer}.  In it, you can type
\emph{extended commands} (invoked by the keystroke M-x) that do not
have keyboard shortcuts.

The minibuffer is also where you get prompted for arguments to various
{\gs} commands.  In those cases, you usually have a choice between
clicking on a visual representation of the argument you would like to
supply as mentioned in section \ref{sec-score-pane}, typing some
textual representation of it directly at the prompt, or using the
right mouse button to get a menu of all possibilities from which you
can select the one you want. 

When typing some textual representation of some existing object, such
as the name of a layer or of a staff, you can usually use
\emph{completion}\index{completion}, which means that you can type a
unique prefix of the text and then use the \kbd{TAB} key to get {\gs}
to fill in the rest.  Using completion after having been prompted for
some argument is usually faster (provided you have some idea of what
the textual representation is) than to grab the mouse and click on the
object in the score pane.

\subsection{Other panes}

In the upper-right corner, you will see a the pane that determines the
\emph{input state}\index{input state}\index{pane!input state|)} The
input state shows a stylized version of a notehead, possibly with a
stem, augmentation dots, and beaming information.  If the stem is
down, all the new clusters (a cluster is rougly the same thing as a
chord) entered will have their stems down.  If the stem is up, all the
new clusters entered will have their stems up.  If the input state
shows both a stem up and a stem down, all new clusters will have a
stem direction of \emph{auto}, indicating that {\gs} will determine
stem direction based on position of the notes on the staff and beaming
information.

In the lower-right corner, you will see a pane that gives more detail
about \emph{current element}\index{current
  element}\index{element!current|)}, i.e., the element (if any)
immediately to the left of the cursor), namely the \emph{current
  element pane}\index{current element pane}\index{pane!current
  element|)}. (We may eliminate this pane in the
future if we can figure out a way to display this information in a not
to disturbing way directly in the score pane).

\section{The cursor}

In the score pane, you will see a yellow vertical line across the
single staff on display and two parallel vertical red lines that are
usually near the yellow line.  This is the {\gs}
\emph{cursor}\index{cursor} where all editing takes place.

The cursor is either before all the elements, after all the elements,
or between two element.  Newly created elements will be inserted
immediately to the left of the cursor (which will make the newly
inserted element the \emph{current} one). 

The yellow line indicates the staff to which newly inserted elements
will belong.  The red lines indicate the interval of pitches that will
be used for newly inserted notes.  This interval is centered around
the last note that was inserted.  You can change this interval by
moving it up or down one octave using the commands \kbd{L} (Lower) and
\kbd{H} (Higher). 

\chapter{Basic editing and cursor motion}

\section{Entering single-note elements}

To enter single-note elements, you use the keyboard and type names of
notes such as \kbd{c}, \kbd{d}, \kbd{e}, \kbd{f}, \kbd{g}, \kbd{a}, or
\kbd{b} (in lower case).  Each time you type the name of a note, a new
single-note element will be inserted to the left of the cursor.

{\gs} will try to guess which octave you want from the previous note
you typed, so that the pitch distance between the two is minimized.
If this choice is not right, you can type a capital \kbd{L} (for
\command{Lower}) in order to tell {\gs} to decrease its idea of what
notes to enter next by one octave, or \kbd{H} (for \command{Higher})
in order to tell {\gs} to increase its idea of what notes to enter
next by one octave.

The visible cursor will indicate what interval {\gs} thinks is the
best for the next inserted note with two parallel vertical red lines. 

\section{Adding notes to existing elements}

In order to add a new note to the current element, you type the note
name with a capital letter such as \kbd{C}, \kbd{D}, \kbd{E}, \kbd{F},
\kbd{G}, \kbd{A}, or \kbd{B}.

Again, {\gs} will try to guess which octave you want from the previous
note you typed, so that the pitch distance between the two is
minimized.

Currently, there is no way to remove a single note from an element
(though we may implement that in the future).  For now, if you add a
note by mistake, the only way you can fix it is by erasing the entire
element and start over. 

\section{Adding a rest}

A rest is another kind of element.  You enter a rest by typing \kbd{,}
(\command{Insert Rest}).  Conceptually, every rest has a notehead as
well as rbeams and lbeams in addition to augmentation dots.  These are
used to determine the duration and what kind of rest to display.

\section{Deleting elements}

To erase the current element, type \kbd{Control-h} (\command{Erase
Element}), i.e. hold the \kbd{control} key down while hitting the
\kbd{h} key.

It is also possible to delete the next element after the cursor by
typing \kbd{Control-d} (\command{Delete Element}). 

In general, we speak about \emph{erasing}\index{erasing} when the
material removed is to the left of the cursor, and
\emph{deleting}\index{deleting} when it is to the right. 

\section{Moving by element}

To move backward and forward over the elements that you have typed,
you can type \kbd{Control-b} (for \command{Backward Element}) or
\kbd{Control-f} (\command{Forward Element}). 

\section{Adding and removing augmentation dots}

If you would like to add an augmentation dot to the current element,
or more generally, to add another dot to the ones already there, you
type \kbd{.} (\command{More Dots}).  To erase one of the dots, type
\kbd{x.} (\command{Fewer Dots}).

The key \kbd{x} can be used in many other situations as a prefix to
indicate the opposite command from the one indicated without the
\kbd{x} prefix.

\section{Adding and removing beaming and flags}

{\gs} maintains beaming information for each element in the following
way.  Each element has a number of right beams (or \emph{rbeams} for
short) and a number of left beams (or \emph{lbeams} for short). 

If the element has neither lbeams nor rbeams, the element is
considered as having neither flags nor beams.  Its duration as far as
layout is concerned will then be determined by the notehead and the
augmentation dots.  The duration of the element will be determined by
the maximum number of lbeams and rbeams.  As usual, the value of the
notehead is divided by two for each beam (and then increased by any
augmentation dots).

If the element has rbeams or lbeams, but neither the element to its
left has any rbeams nor does the element to its right have any lbeams,
the element will be displayed with a number of flags corresponding to
the maximum of its lbeams and rbeams.

If the element has lbeams and the element to the left has rbeams, then
a beam will be displayed between the two elements.  The number of
beams that will be displayed is the minimum of the rbeams of the
element to the left and the lbeams of the element itself.  Remaining
beams are displayed as fractional beams \unimp{We only show a single
  beam at the moment}.  The symmetric rule holds for the element to
the right. 

To increase the number of rbeams of the current element, type \kbd{]}
  (\command{More Rbeams}).  To decrease the number or rbeams of the
  current element, type \kbd{x]} (\command{Fewer Rbeams}).  The
  corresponding commands for the lbeams are \kbd{[} (\command{More
  Lbeams}) and \kbd{x[} (\command{Fewer Lbeams}). 

Thus, to obtain a two-element beam group, the first with a C and the
second with a D, you would type the sequence \kbd{c]d[} (think of it
as ``c, beam right, d, beam left'').  

\section{Changing the notehead}

In order to change the notehead of the current element, you type
\kbd{Meta-h} (\command{Rotate Notehead}).  If the notehead of the current
element is \emph{whole}, it will change to \emph{half}, if it is
\emph{half}, it will change to \emph{filled}, and if it is
\emph{filled}, it will change to \emph{whole}.  

When the notehead is \emph{whole}, no stem, flags or beams will be
displayed, nor taken into account to compute the duration of the
element.  The information will not be removed if present, however, so
that if you change the notehead from \emph{filled} to \emph{whole} and
then back to \emph{filled}, you will be back to the original element. 

When the notehead is \emph{half}, a stem will be displayed, but no
beams or flags.  Again, the information is not erased, in case you
would like to change back to a \emph{filled} notehead. 

\chapter{Changing individual notes in an element}

{\gs} maintains a concept of a \emph{current note}\index{current
  note}\index{note!current|)} of the current cluster.  A note that has
just been inserted or added to an existing element becomes the
current note.  Several commands act upon the current note. 

To add a pitch alteration to the current note, you can type \kbd{\#}
(\command{sharper}) to make the key sharper (if you repeat this
command again, it will become double sharp) or \kbd{@}
(\command{flatter}).  Whether or not this operation results in an
accidental being displayed, depends on the key signature of the staff.
You can change the key signature of the staff by using the command
\kbd{Meta-\#} (More Sharps) to add another sharp sign to (or to remove
a flat sign from) the signature or the command \kbd{Meta-@} (More
Flats) to add another flat sign to (or to remove a sharp sign from)
the signature. 

To change the pitch of the current note, you can type \kbd{Meta-u}
(Up) to increase it or \kbd{Meta-d} (Down) to decrease it.  Notice
that {\gs} does not allow for two different notes having the same
pitch in a cluster.  An attempt to move the pitch of the current note
so that it ends up on top of an existing one, will remove one of the
notes.

The commands \kbd{Meta-u} and \kbd{Meta-d} can also be used on a rest,
in which case they change the vertical staff position of the rest.
This is useful if you have several voices on one staff and you would
like to have one voice higher on the staff than the other.  In the
future, we will implement the possibility for a voice to have a
default staff line for rests. 

In a multi-notehead cluster, you will have to change the current
note.  You can do this by typing \kbd{p} to make the note with the next
higher pitch the current one, or \kbd{n} to make the note with the
next lower pitch the current one.  

Occasionally, you might want to display an accidental even though the
layout engine decided it was not necessary because the staff position
of the note has the same pitch alteration inherited from the key
signature.  \unimp{This is a matter of adding another slot to a
note}. 

\chapter{Changing the input state}

If you have a long sequence of elements, each with (say) three flags
to type, you would not want to enter these flags for each element,
since you would have to type something like
\kbd{c]]]d]]]e]]]f]]]g]]]}.  A better idea would be to change the
input state (or istate for short) so that each new element has three
rbeams and then just to type \kbd{cdefg}.

Changes to the input state are made in the score pane, but they have
an immediate effect on the presentation in the input state pane. 

Commands to change input state mimic those of changes to the current
element, except prefixed by \kbd{i}.  Thus, to add in rbeam to the
input state, you would type \kbd{i]}, and to erase one rbeam you would
  type \kbd{ix]}.  Adding and removing lbeams and dots work in the
    analogous way. 

Similarly, in order to change the notehead of the input state, you
type \kbd{in} (\command{Istate Rotate Notehead}).  If the notehead of
the input state is \emph{whole}, it will change to \emph{half}, if
it is \emph{half}, it will change to \emph{filled}, and if it is
\emph{filled}, it will change to \emph{whole}.

\chapter{Operations on measures}

You add a barline at the cursor location by typing \kbd{|}.  As
with other element, the cursor will be positioned right after the
newly created barline.  In fact, a barline is just like any
other element as far as editing operations are concerned.

Thus, to delete a barline, you simply use the ordinary commands
for erasing or deleting elements.

\chapter{Operations on staves}
\label{chap-op-staves}

Initially, when {\gs} starts up, there is a single staff on display.
It is a standard five-line staff with a treble clef on staff step 2
(we count the bottom line of the staff as staff step 0).

\section{Renaming and existing staff}

Staves have unique names in {\gs}.  Initially, the name of the first
staff on display is some default name such as ``default staff''.  You
typically want to use names of staves that suggest the kind of musical
material displayed on it, such as ``Trumpets'' or ``Piano right
hand'', etc. 

You can give a different name to an existing staff by issuing the
command \command{Rename Staff}, either from a menu or form the
minibuffer.  This command prompts for a staff to rename and a new name
for the staff.  At the prompt for the staff to rename, you have a
choice of clicking on a staff on display as indicated in section
\ref{sec-score-pane}, or typing its name in the minibuffer possibly
with \emph{completion}\index{completion} as indicated in section
\ref{sec-score-pane}.  Renaming a staff is such an infrequent
operation that there is no keyboard shortcut for it.

\section{Inserting a new staff}
\label{sec-inserting-staff}

Unless you are writing some very simple music, or a part for a
monophonic instrument, you probably want to be able to add new staves
to the score. Because adding new staves are done fairly infrequently,
there is no keyboard shortcut for doing it.  Instead, you have to
issue one of the commands for doing this either by typing it (with
completion, see \ref{sec-minibuffer}) in the minibuffer or using
the mouse to select one from a menu.  

{\gs} imposes an \emph{order} on the staves of a score.  This order is
the one used to display the staves from top to bottom.  There are two
commands available for inserting new staves, \command{Insert Staff
  Below} to insert a staff \emph{below} an existing one, and \command{Insert
  Staff Above} to insert a staff \emph{above} an existing one. 

In both cases, you will be prompted for some required information, in
order for {\gs} to be able to create the new staff.  First, you will
be prompted for an existing staff below which (in the case of
\command{Insert Staff Below} or above which (in the case of
\command{Insert Staff Above} you would like to insert the new staff.
As usual, you can either click on a visual representation of the staff
in the score pane (see section \ref{sec-score-pane}) or type its name
in the minibuffer with completion (see section
\ref{sec-minibuffer}). 

Next, you are prompted for the type of the staff to create.  There are
currently two types of staves, namely `fiveline' and `lyrics'.  At the
moment, the only way to answer this question is to type it in the
minibuffer (again, completion is available). 

If you requested a five-line staff to be created, you will also be
prompted for the type of clef you would like the staff to have.  There
are three possible choices `treble', `bass', and `c', which you have
to type in the minibuffer at the prompt.  You will also be prompted
for a `line' number on which the clef is to be placed.  Recall that
the lines are numbered with even numbers starting with `0' for the
bottom line of the staff.  The normal place for a treble clef is thus
`2' and the normal place for a bass clef is `6'.  For the c clef it
varies. 

If instead you requested a lyrics staff to be created, there is no
more information to supply.

\section{Deleting a staff}

To delete an existing staff, you issue the \command{Delete Staff}
command, either from a menu or in the minibuffer.  Deleting an
existing staff is such an infrequent operation that no keyboard
shortcut is provided. 

The command prompts for a staff to be deleted.  As usual, you can
either satisfy the request by clicking on the visual representation of
a staff in the score pane (see section \ref{sec-score-pane}) or typing
a response in the minibuffer (see section \ref{sec-minibuffer}). 

\section{Changing the key signature}

To alter the key signature of a staff, use the commands \kbd{Meta-\#}
\command{More Sharps} and \kbd{Meta-@} \command{More Flats}. 

\chapter{Operations on layers}

{\gs} organizes musical material into \emph{layers}\index{layer}.  A
{\gs} layer corresponds roughly to a \emph{voice}\index{voice} in
traditional music terminology.  The reason a different term was chosen
is that there might be some notations that require a voice to be split
into several layers.  Each layer can be displayed on one or more
staves, one of which is always the \emph{current}\index{current staff}
one. 

In {\gs}, layers have unique names.  Initially, when {\gs} starts up,
it has a single layer with some default name such as ``default
layer''. It has a single staff on which it is displayed, namely the
default staff (see chapter \ref{chap-op-staves}).  

You typically want to choose names for layers that suggest the kind of
music material that is contained in it.  Since a layer has (or will
have) a unique \emph{instrument}\index{instrument} associated with it,
we advice you not to mix material for different instruments in a
layer.  Therefore, when a part plays several instruments (presumably
not simultaneously) we advice you to use a different layer for each
instrument in the part.  Typically, then, a layer name would be the
name of an instruments playing a single part, such as ``First Violin''
or ``Solo Guitar''.  

\section{Renaming an existing layer}

You can rename any layer by issuing the command \command{Rename
Layer}, either in the minibuffer (see section
\ref{sec-minibuffer}) or from a menu.

You will first be prompted for a layer to rename.  Currently the only
way to satisfy this request is by typing the name of the layer to the
prompt in the minibuffer. Completion is possible as usual (see
section \ref{sec-minibuffer}). 
 
Next, you will be prompted for a new name of the layer.  To satisfy
the request, you type any string at the prompts.  Notice that names
must be unique.  If you type the name of an existing layer, your
entire command will be rejected. 

\section{Selecting a layer}

{\gs} has the concept of a \emph{current layer}\index{current layer}
which receives all music material typed in the score pane, such as
notes, rests, etc. 

To change the current layer, issue the \command{Select Layer} either
from a menu or in the minibuffer.  

You will be prompted for a layer to be used as the current one.  At
the moment, the only way to satisfy this request is to type its unique
name (with completion, see \ref{sec-minibuffer}) at the prompt in
the minibuffer. 

\section{Adding a new layer}

Before adding a new layer, make sure you have created one of the
staves on which you would like for its material to be displayed (see
section \ref{sec-inserting-staff}).

Adding a new layer is done by issuing the command \command{Add
  Layer}.  You will be prompted for a unique name of the new layer,
and for an existing staff to use as the initial staff of the layer. 

The type of the layer that is created as a result of this command
depends on the type of the initial staff.  If the initial staff is an
ordinary five-line staff, then a melody layer is created.  If instead
the initial staff is a lyrics staff, a lyrics layer is created. 

\section{Deleting a layer}

To delete the current layer, you have to use the minibuffer (since
this is an operation that is presumably rare).  The command to use is
\command{Delete Layer}. You will be prompted for a layer to delete. 

\section{Adding an existing staff to a layer}

A layer can be displayed on one or several \emph{staves}.  Most layers
will have a single staff associated with them, but is occasionally
useful to have two or more.  Clusters within such a layer with notes
in different staves will have their stems span those staves.  At other
times, it is useful simply to have some elements of the layer
displayed on one staff and some other elements on a different
staff.

To add a staff to the possible staves of a layer, use the command
\command{Add Staff To Layer}.  You will be prompted for a staff to add
and a layer to add it to.  

\section{Stem direction of a layer}

A layer can have a preferred stem direction for all clusters in
it\unimp{It is not hard to do, though}.  The user can override this
stem direction for individual elements, but if the stem direction of
the cluster is \emph{auto}\index{auto!stem direction|)}, then it will
be displayed with the stem direction determined by the layer.  If the
stem direction of the layer is also \emph{auto}, the layout engine
will determine the stem direction. 

% ***********************************************************
\part{Reference manual}
\inputtex{model.tex}
\inputtex{commands.tex}

% ***********************************************************
\part{Internals}

%======================================================
\chapter{Implementation of the user model}

The implementation of the user model pretty much reflects the model
itself, except that the implementation of the \emph{slice} is not a
simple sequence of elements.

Instead, a \emph{slice} is a sequence of a user-invisible type called
a \emph{bar}.  We introduced this additional type to simplify the
layout algorithm.  A bar contains all the non-barline elements
between two barlines.  The layout algorithm therefore does not
have to start by finding the barlines in the sequence, but can
directly line up the material between barlines.

The implementation of the model is actually spread out in three
different files:

\begin{itemize} 
\item \texttt{buffer.lisp}: This file is the one that most immediately
  corresponds to the user model.  Generic functions allow insertion
  and deletion of the different types used to implement the model.
  Currently, the implementation of various sequences is in the form of
  simple Lisp lists.  We might change that one day for performance
  reasons, but then again, it may not be necessary. 
\item \texttt{numbering.lisp}: This file observes the action (through
  :before, :after, and :around methods) of the basic operations in
  order to number different instances of the basic types.  This allows
  us to know at any point in time whether an instance precedes another
  in a sequence.  The implementation of the numbering scheme is not
  very efficient at the moment, but it might not matter since
  sequences are fairly short in general.
\item \texttt{cursor.lisp}: This file implements cursor operations.
  It observes operations in the preceding two files in order to move
  around cursors.  
\end{itemize}

A cursor has two slots, a bar in which it is located, and the position
within the bar.  Cursors are inserted on most of the model types (bar,
slice, segment, buffer) so that when one of those elements is deleted,
the cursor can be moved accordingly.  Things get complicated because
we want to allow for an arbitrary number of cursors, each attached to
a different window, though we do not currently take advantage of that
possibility.

%======================================================
\inputtex{buffer.tex}
%======================================================
\chapter{The Layout algorithm}

\section{Purpose}

The purpose of the layout algorithm is to divide the entire score into
displayed units.  Currently, these units are \emph{lines}, where a
line is the part of system displayed in sequence from left to right.
In the future, it will be possible to use \emph{pages} as units which
will make things a bit more realistic.  Since we currently do not have
pages, our display is mostly in the form of a ``roll of paper''.

\section{What it needs to recompute}

The layout algorithm is invoked after each keystroke.  But in order to
do its work correctly, it has to know what has been damaged between
two invocations.  The user can have executed some arbitrary code
between two such invocations, and in order not to have to recompute
the entire score each time, it preserves the result of its calculation
between invocations and only recompute what is necessary. 

To know what has been damaged, the layout algorithm (in the file
\texttt{measure.lisp}) installs a different set of \texttt{:before},
\texttt{:after}, and \texttt{:around} methods to observe the
operations on the fundamental implementation of the user model.
Currently, it does its work in a very crude way.  Whenever the
structure has been modified in any way, the entire segment is marked
as modified and will be totally recomputed at the next invocation. 

Recomputing the layout involves computing, for each segment, a
sequence of \emph{measures}\index{measure}.  This is a user-invisible
type that corresponds to the superimposition of simultaneous bars in
overlapping layers.  We could be more careful about marking things as
damaged, and preserve measures that contain bars that have not been
modified.  We might do that if performance requires it.  At the
moment, we have no reason to believe that it will be necessary.  

\section{The importance of segments}

The importance of the concept of a segment should now be clear.
If the user deletes a barline, all measures to the right of the
point of deletion \emph{within the same segment} must be recomputed,
since the musical material will no longer line up the same way.  Had
the score been a single segment, we would have had to potentially
recompute the entire score in such situations.  In addition, it would
be pretty annoying to the user not to be able to count on finished
material far from the cursor to remain intact.  Thus, the concept of a
segment both helps the user organize his material and helps the
implementor limit the damage that needs to be repaired by the layout
algorithm. 

\section{Head and tail slices}

There is a slight problem with \emph{head slices} and \emph{tail
slices} which is why they are not implemented yet.  While
conceptually these slices belong to the segment of their
corresponding layer, the layout algorithm must align them with the
last measure of the previous segment and the first measure of the
next segment.  To make it impossible for those slices to stick out
arbitrarily far, we do not allow for head and tail slices to contain
barlines.  This limits the stick-out to the previous and the
next segments.  We have not yet decided how to manage these slices.
Probably, a modification would have to mark the previous or the next
segment as damaged, or at least the last measure or the first measure
in these segments.  

\section{The concept of a measure}

A measure, as far as the layout algorithm is concerned, only contains
some very limited amount of information about its bars so that it will
know how much room the measure will take up on a given line of musical
material. 

The main problem we need to address is the fact that a measure takes
up a different amount of space according to the contents of the other
measures of the same line.  The remainder of this section will discuss
how we solve that problem.

First, let us introduce the concept of a \emph{time line}\index{time
line}. A time line represents a moment in time in a given measure such
that the measure has some event (like notes) that start at that moment
in time.  The \emph{duration}\index{duration!of a time line|)} of the
time line is the temporal distance between it and the following one in
the measure, or, if it is the last one, the duration of the longest
lasting event that starts at the moment in time. 

From the art of music engraving, we know that a longer temporal
distance requires a longer geographic distance on the score as well,
but that is not all there is to it.  In fact, we also know that
\emph{on a given line}, the smallest geographic distance possible (a
global parameter of the score that determines denseness and that we
will call $w_{min}$) is assigned to the shortest temporal distance on
that line, \emph{independently of the absolute value of that temporal
distance}. The other geographic distances are adjusted accordingly.
Thus, a measure can take up a lot more space on a line if other
measures on the same line have shorter minimal temporal distances
between time lines than itself.

From the articles published about the Lime score editor, we use the
following formula relating the ratio between the geographic distance
of two time lines to the ratio of their durations:

$w_2 / w_1 = {(d_2 / d_1)}^k$

where $w_1$ and $w_2$ are the ``widths'', i.e. geographic distances of
the two time lines, and $d_1$ and $d_2$ are their durations. The
parameter $k$ takes on values between $0$ and $1$, where $0$ gives
constant spacing (space is independent of duration) and $1$ gives
proportional spacing.  A good value for $k$ seems to be around $0.6$. 

On a given line, with a smallest temporal distance $d_{min}$ between
any adjacent time lines, we can express the geographic distance
$w$ following any time line with duration $d$ as:

$w = w_{min} {(d / d_{min})}^k$

Let us now define the \emph{natural width} of a measure as:

$W_{nm} = \sum_i w_{min} {(d_i / d_{min})}^k$

where $d_i$ is the duration of the $i$th time line of the measure, and
the \emph{natural width} of a line $W_{nl}$ as the sum of the natural
widths of the measures of the line.  

We can rewrite the formula above as:

$W_{nm} = {(1 / d_{min})}^k w_{min}  \sum_i {d_i}^k$

Where $w_{min}  \sum_i {d_i}^k$ is a property of a measure that
remains constant as long as the two global parameters $w_{min}$ and
$k$ do not change.  We shall call this property the \emph{measure
coefficient}.  To get the natural width of the measure, it suffices to
multiply its coefficient with ${(1 / d_{min})}^k$, which varies
between different lines.  Thus, we can compute the coefficient for a
measure once and for all (again as long as the global parameters stay
the same). 

To compute the natural width of a line, we have to compute the sum of
the coefficients of the measures on the line and the minimum of their
respective minimal time line durations.  This is a constant time
operation that does not require looping over individual measures of
the line.  

Now that we know how to compute the natural width of a line, we need
to divide the score into lines in such a way that we have the best
possible layout.  For that, we need to define the \emph{cost} of a
line.so that we can compare two different suggested lines.  In {\gs}
we currently use the amount we have to stretch or compress it compared
to its natural width in order for it to fit on a line.  Let us define
the \emph{compress factor} of the line as the quotient between the
natural width of the line and the width of the page.  We then define
the \emph{cost} of a line as the maximum between the compress factor
and the its inverse.  That is not the only way to do it.  We might
imagine penalizing compressions more than stretches, as it might be
much harder to fit the musical material on the line when it needs to
be compressed. 

Our definition of the natural width of a measure does not take into
account extra material such as accidental that need more room.  We
could do a more precise job by adding in some extra room for such
material and count it toward the natural width of the line only if the
measure does not need to be stretched (i.e. only if the minimum
duration of the line is the same as that of the measure).  This would
only slightly complicate the method of combining measure parameters
into line parameters and it might give a considerably better result. 

However, it might not be very important to improve this calculation.
The result of the calculation is only used to divide the score into
lines and pages.  Once we display the few visible pages of the score,
we have great freedom to reorganize the material on the page.  Dense
measures might have to borrow space from less dense ones on the same
line.  As I recall, the article on the Lime music editor describes how
to do this in great detail, although it is not yet implemented in
{\gs}.  Since usually even under very crowded conditions, space can be
reorganized within the line, the only risk we run of not having a very
precise calculation of the natural width of a line is that one line
might look slightly denser than another one on the same page, because
we did not take into account all of its accidentals. 

\section{Breaking measures into lines}

Breaking the sequence of all measures in the score into subsequences
corresponding to lines is done by an algorithm that is similar to
\emph{dynamic programming}.  Essentially, we compute all possible line
breaks and choose the best one, but thanks to dynamic programming we
avoid a possibly exponential complexity.  

But even the cubic complexity of dynamic programming would be
unacceptable for our purposes.  We transform that complexity from
cubic to constant by a series of tricks, which are documented in a
separate (unpublished) article in the same directory as this one. 

%======================================================
\chapter{{\gs} as a {\clim} application}

%======================================================
\chapter{The {\obseq} library}

%======================================================
\chapter{Fonts}
\label{chap-fonts}

In order for {\gs} to have a nice appearance on the screen, it was
determined essential to use anti-aliased fonts.  The problem is that
{\xwin} does not have such fonts in the standard.  There is an
extension to the XFree86 server for {\xwin} which uses anti-aliased
fonts, but it is not an official standard for {\xwin}.  In addition,
this extension requires the \emph{client} to convert glyphs to
rasters, which traditionally has been the responsibility of the server
of {\xwin}.  Gilbert Baumann has created a new server for {\xwin} that
is capable of anti-aliased fonts, but again, this is not available in
all servers for {\xwin}.

As mentioned above, in recent extensions to {\xwin}, converting the
glyph to a raster is a client-side operation.  Usually, this operation
is done by a library such as {\freetype}, which can use font of type
TrueType or PostScript Type 1.  Each of these systems has its own
method for rendering low-resolution fonts.  

I decided that if converting glyphs to rasters were to be a
client-side operation, I might as well have full control over it, and
use the {\metafont} program for this purpose.  {\metafont} is a
universal programming language that does not suffer from the
limitations of the \emph{hints}\index{hints} in systems such as
PostScript Type 1.  Using {\metafont}, it would even be possible to
create fonts specifically useful for anti-aliasing.

The font used by {\gs} (called SDL, for Score Drawing Library), is
created by a {\metafont} program that generates glyphs that are
magnified a factor four.  The resulting \texttt{gf} file is parsed by
a {\commonlisp} library that creates a {\clim} pixmap which contains an
anti-aliased version of the glyph.  

The difference between this solution and the (more general) one based
on a library such as {\freetype}, is that {\freetype} (together with
the \emph{render}-extension of {\xwin}) provides \emph{transparency}
(also known as an \emph{alpha}-channel), which greatly simplifies
high-quality output, whereas using pixmaps makes it necessary to
display musical output in a certain order, and to do a certain amount
of cheating.

The great advantage of the {\gs} solution, though, is that it will
work with any basic server for {\xwin}, and that it is very fast. 

The main problem with our solution (and this is why cheating is
necessary) is that glyphs can partly overlap on the page.  In
particular, this occurs in stacks of noteheads that may or may not
belong to the same cluster (they could come from different layers,
only be placed a staff step apart), but also trivially between any
music character and the staff lines.  When that happens with ordinary
fonts, no problem results, but when gray-scale fonts are used, they do
not combine very well.  

For staff lines, stems, and ledger lines, we solve the problem easily
by imposing that these characters always be pixel aligned (and thus
completely black).  It then suffices to draw these characters
\emph{after} the ones they overlap with.  

For stacks of noteheads, the problem is greater.  We solve it by
substituting different glyphs that correspond to combinations of
noteheads in a stack.  To avoid having an unbounded number of such
combinations, we have one glyph for the lower half of a notehead, one
for the upper half, and one for the combination of the upper half of
one notehead and the lower half of the one on top of it.  All these
glyphs are properly anti-aliased and their bounding boxes do not
overlap so combining them does not create any problems \unimp{We still
need to use the {\clim} output recording mechanism to remember
positions of noteheads and to substitute these alternative glyphs}. 

The way we are planning to substitute alternative glyphs is by using
the \emph{output recording} mechanism of {\clim}.  This mechanism is
responsible for repainting windows that have had their contents
partially lost due to being covered or invisible.  It work by
capturing output and storing output records in the pane.  We plan to
use it to introduce a new kind of output record that guarantees that
repainting  will be done in the right order, and that alternative
glyphs will be substituted. 

%======================================================
\chapter{Beam drawing}

It might seem like drawing beams would be trivial.  {\xwin} provides
adequate primitives to draw filled polygons that would be
generalizations of beams in a score.  As with fonts (see chapter
\ref{chap-fonts}) the problem is that anti-aliasing is called for in
order for the visual appearance to be acceptable, and that {\xwin}
does not provide for anti-aliasing in the basic protocol.  

A possibility would be to treat beams as any other character glyphs
and to write a {\metafont} program for creating beams.  The problem
with this solution is that there is a large number of different beam
slopes possible, and that a font would have to include a very large
number of them.  Furthermore, beams having fairly small slopes, turn
out to use only a small fraction of the 16 different gray-levels that
we would like to use for anti-aliasing. 

For that reason, we generate beams on an as-needed basis.  A beam is
divided into segments, each with a $\Delta y$ of $1$ (but each
possibly with a different $\Delta x$.  Each segment is drawn as a
central black rectangle surrounded by two pixmaps, each of height $1$
and the length of the segment.  One of the pixmaps provides
gray-levels from black to white and the other one from white to
black.  This method gives a smoother impression than if polygon
drawing were used.  Each new segment is similar to the previous one
(its length might vary a little) and positioned one position higher or
lower than the previous one.

% ***********************************************************
\part{Appendices}
\appendix

\inputtex{history.tex}
\inputtex{beaming-algo.tex}
\inputtex{accidentals.tex}
\inputtex{release-notes.tex}
\inputtex{plans.tex}

%======================================================
\chapter{{\clim} terminology}
\label{clim-terminology}

Talk about frames, panes, streams, etc.

\nocite{ross-1987} \nocite{haken-1993} \nocite{haken-1995}
\nocite{aho-hopcroft-ullman} \nocite{blostein-1991}
\nocite{blostein-1994} \nocite{gourlay-1987-spacing}
\nocite{gourlay-1987-formatting} \nocite{hegazy-1987}
\nocite{hegazy-1987-breaking} \nocite{rader-1996} \nocite{tex}


\printindex

\newpage
\bibliographystyle{alpha}
\bibliography{gsharp}
\end{document}
