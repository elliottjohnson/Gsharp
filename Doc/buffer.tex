\chapter{Buffer protocols}

%===================================================================
\section{The clef protocol}

%-------------------------------------------------------------------
\subsection{Description}

A \emph{clef}\index{clef} is an object associated with a \emph{staff},
indicating where (on which staff step) notes are to be rendered on the
staff.  The \emph{name}\index{clef!name of|}\index{name!of a clef|} of
the clef is one of \lispobj{:treble}, \lispobj{:bass}, \lispobj{:c},
and \lispobj{:percussion}.  The \emph{line number}\index{clef!line
  number of|}\index{line number!of a clef|} of the clef indicates the
staff line on which the clef should be rendered. 

The name of the clef (except \lispobj{:percussion}) corresponds to a
certain pitch.  Notes of that pitch will be rendered on the line of
the clef.  Each different type of clef (as indicated by the name) has
a line by default which is assigned to the clef at creation if no
argument is given

Here is a table of pitches and default line numbers of the different
clefs:

\begin{tabular}{|l|l|l|}
\hline
Name                   & Pitch        & Default line \\
\hline\hline
\lispobj{:treble}      & G (octave??) & 2 \\
\lispobj{:bass}        & F (octave??) & 6 \\
\lispobj{:c}           & C (octave??) & 4 \\
\lispobj{:percussion}  & -            & 3 \\
\hline
\end{tabular}

%-------------------------------------------------------------------
\subsection{Protocol classes and functions}

\Defclass {clef}

\Defun {make-clef} {name \optional (lineno \cl{nil})}

Create a clef with the name and line numbers given.  Line numbers
default to the values in the table above. 

\Defgeneric {name} {clef}

Return the name of the clef given as argument.

\Defgeneric {lineno} {clef}

Return the line number of the clef given as argument. 

%-------------------------------------------------------------------
\subsection{External representation}

A clef is printed (by \lispobj{print-object}) like this in version 3
of the external representation :

\texttt{[K :name \textit{name} :lineno \textit{lineno} ]}

The reader accepts this syntax, except that the slots can come in any
arbitrary order. 

%===================================================================
\section{The staff protocol}

%-------------------------------------------------------------------
\subsection{Description}

%-------------------------------------------------------------------
\subsection{Protocol classes and functions}

\Defclass {staff}

The protocol class for all staves. 

\Defclass {fiveline-staff}

This class is a subclass of \lispobj{staff}, and is used to represent an
ordinary five-line staff for displaying notes. 

\Defgeneric {name} {staff}

Return the name of the staff.  With \lispobj{setf}, change the name of
the staff.

\Defun {make-fiveline-staff} {\key name clef keysig}

Make a five-line staff with the name, the clef, and the key signature
indicated.  The default value of the \lispobj{name} argument is the
string \lispobj{"default staff"}.  The default value for the
\lispobj{clef} argument is a newly created treble clef.  The default
argument for the key signature is a key signature with no
alterations. 

\Defgeneric {clef} {fiveline-staff}

Return the clef of the staff.  With \lispobj{setf}, change the clef of
the staff.

\Defgeneric {keysig} {fiveline-staff}

Return the key signature of the staff.  With \lispobj{setf}, change
the key signature of the staff. 

\Defclass {lyrics-staff}

This class is a subclass of \lispobj{staff}, and is used to represent a
staff for displaying lyrics. 

\Defun {make-lyrics-staff} (\key name)

Make a lyrics staff with the name indicated.  The default value of the
\lispobj{name} argument is the string \lispobj{"default staff"}.

%-------------------------------------------------------------------
\subsection{External representation}

A fiveline staff is printed (by \lispobj{print-object}) like this in
version 3 of the external representation:

\texttt{[= :name \textit{name} :clef \textit{clef} :keysig \textit{keysig} ]}

The reader accepts this syntax, except that the slots can come in any
arbitrary order. 

A lyrics staff is printed (by \lispobj{print-object}) like this in
version 3 of the external representation:

\texttt{[L :name \textit{name} ]}

%===================================================================
\section{The keysig protocol}

There is no keysig protocol yet.  But I would like to create one.  A
keysig would be a read-only object.  

%===================================================================
\section{The note protocol}

%-------------------------------------------------------------------
\subsection{Description}

Notes are immutable objects.  For that reason, if you want to change
some characteristics of a note in a cluster, you have to delete the
note from the cluster and create one with the characteristics you
would like. 

%-------------------------------------------------------------------
\subsection{Protocol classes and functions}

\Defclass{note}

The protocol class for notes. 

\Defun {make-note} (pitch staff \rest args \key head accidentals dots)

Create a note as indicated.  The pitch represents the pitch of the
note without any accidentals, and is an integer between 0 and 127,
where 0 indicates a C in the lowest octave.  The staff is an instance
of the class \lispobj{staff} and indicates what staff the note is to
be displayed on.  

The \lispobj{head} argument indicates what kind of notehead is wanted
in the form of one of the keywords \lispobj{:whole}, \lispobj{:half},
and \lispobj{:filled}, or nil, where nil means that the notehead is
taken from the cluster to which the note belongs.

The \lispobj{accidentals} argument indicates which, if any,
accidentals the note should have, and is one of the keywords 
\lispobj{:natural}, \lispobj{:flat}, \lispobj{:double-flat},
\lispobj{:sharp}, or \lispobj{:double-sharp}.  The default value is
\lispobj{:natural} meaning that this note does not have any
accidentals. 

The \lispobj{dots} argument indicates how many dots follow this note.
It is an integer between 0 and 3 or nil, where nil means that the
number of dots is taken  from the cluster to which the note belongs.

%-------------------------------------------------------------------
\subsection{External representation}

A note is printed (by \lispobj{print-object}

%===================================================================
\section{The cluster protocol}

%-------------------------------------------------------------------
\subsection{Description}

A cluster is roughly the same as a chord, in that it is a collection
of notes that share a stem (unless the noteheads are whole noteheads,
in which case there is no stem at all).

%-------------------------------------------------------------------
\subsection{Protocol classes and functions}

\Defclass{cluster}

\Defun {make-cluster} {\key (notehead :filled) (lbeams 0) (rbeams 0)
		       (dots 0) (xoffset 0) notes (stem-direction :auto))}

Create a cluster.

The \lispobj{notehead} argument indicates the basic timing unit to be
used for this cluster and is one of the keywords \lispobj{:whole},
\lispobj{:half}, and \lispobj{:filled}.  A value of \lispobj{:whole}
gives a \emph{basic duration} of $1$, a value of \lispobj{:half} gives $1
\over 2$ and \lispobj{:filled} gives $1 \over 4$. 

The \lispobj{lbeams} argument indicates by how many beams this cluster
ties to the cluster immediately to its left.  It must be an integer
between $0$ and $5$ inclusive. 

The \lispobj{rbeams} argument indicates by how many beams this cluster
ties to the cluster immediately to its right.  It must be an integer
between $0$ and $5$ inclusive. 

When the \lispobj{notehead} argument is \lispobj{:filled}, the
\emph{basic duration} ($1 \over 4$) is divided by $2^n$ where $n$ is the max
of the values of \lispobj{lbeams} and \lispobj{rbeams}, giving the
\emph{beamed duration} of the cluster. 

The \lispobj{dots} arguments indicates how many dots should follow
notes of this cluster by default.  It must be an integer between $0$
and $3$ inclusive.  The number of dots determines how the \emph{final
duration} for the cluster is obtained from its \emph{beamed duration}.
When the value is $0$, the final duration is the same as the beamed
duration.  When the value is $1$, the final duration is the beamed
duration multiplied by $3 \over 2$.  When the value is $2$, the final
duration is the beamed duration multiplied by $7 \over 4$.  When the
value is $3$, the final duration is the beamed duration multiplied by
$15 \over 8$.

The \lispobj{xoffset} arguments indicates by how much to the right of
its nominal position this cluster should be displayed.  The unit is in
\emph{staff-steps}. 

The \lispobj{notes} argument is a list of initial notes for this
cluster.

The \lispobj{stem-direction} argument indicates the direction to be
used for the stem of this cluster.  The value is one of the keywords
\lispobj{:up}, \lispobj{:down} and \lispobj{:auto}, where
\lispobj{:auto} means that the direction is determined by the layout
algorithm.  The values \lispobj{:up} and \lispobj{:down} are typically
used when two voices share a staff, and one voice has all the stems up
and the other has all the stems down. 

\Defun {make-rest} {staff \key (staff-pos 4) (notehead :filled) (lbeams 0) (rbeams 0)
		    (dots 0) (xoffset 0)}

\Defun {make-lyrics-element} {staff \key (notehead :filled) (lbeams 0) (rbeams 0)
		    (dots 0) (xoffset 0)}

%-------------------------------------------------------------------
\subsection{External representation}


%===================================================================
\section{The bar protocol}

%-------------------------------------------------------------------
\subsection{Description}


%-------------------------------------------------------------------
\subsection{Protocol classes and functions}

\Defclass{bar}

\Defclass{melody-bar}

\Defun {make-melody-bar} {\key elements}

\Defclass{lyrics-bar}

\Defun {make-lyrics-bar} {\key elements}

%-------------------------------------------------------------------
\subsection{External representation}


%===================================================================
\section{The slice protocol}

%-------------------------------------------------------------------
\subsection{Description}


%-------------------------------------------------------------------
\subsection{Protocol classes and functions}

\Defclass{slice}

\Defun {make-slice} {\key bars}

%-------------------------------------------------------------------
\subsection{External representation}


