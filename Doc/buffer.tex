\chapter{Buffer protocols}

%===================================================================
\section{The clef protocol}

%-------------------------------------------------------------------
\subsection{Description}

A \emph{clef}\index{clef} is an object associated with a \emph{staff},
indicating where (on which staff step) notes are to be rendered on the
staff.  The \emph{name}\index{clef!name of|}\index{name!of a clef|} of
the clef is one of \lispobj{:treble}, \lispobj{:bass}, \lispobj{:c},
and \lispobj{:percussion}.  The \emph{line number}\index{clef!line
  number of|}\index{line number!of a clef|} of the clef indicates what
staff line on which the clef should be rendered. 

The name of the clef (except \lispobj{:percussion}) corresponds to a
certain pitch.  Notes of that pitch will be rendered on the line of
the clef.  Each different type of clef (as indicated by the name) has
a line by default which is assigned to the clef at creation if no
argument is given

Here is a table of pitches and default line numbers of the different
clefs:

\begin{tabular}{|l|l|l|}
\hline
Name                   & Pitch        & Default line \\
\hline\hline
\lispobj{:treble}      & G (octave??) & 2 \\
\lispobj{:bass}        & F (octave??) & 6 \\
\lispobj{:c}           & C (octave??) & 4 \\
\lispobj{:percussion}  & -            & 3 \\
\hline
\end{tabular}

%-------------------------------------------------------------------
\subsection{Protocol classes and functions}

\Defclass {clef}

\Definitarg {:name}

\Definitarg {:lineno}

\Defun {make-clef} {name \optional (lineno \cl{nil})}

Create a clef with the name and line numbers given.  Line numbers
default to the values in the table above. 

\Defgeneric {name} {clef}

Return the name of the clef given as argument.

\Defgeneric {lineno} {clef}

Return the line number of the clef given as argument. 

%-------------------------------------------------------------------
\subsection{External representation}

A clef is printed (by \lispobj{print-object}) like this in version 3
of the external representation :

\texttt{[K :name \textit{name} :lineno \textit{lineno} ]}

The reader accepts this syntax, except that the slots can come in any
arbitrary order. 

In version 2 of the external representation, a clef was written like
this :

\texttt{[K \textit{name} \textit{lineno} ]}

%===================================================================
\section{The staff protocol}

%-------------------------------------------------------------------
\subsection{Description}

%-------------------------------------------------------------------
\subsection{Protocol classes and functions}

\Defclass {staff}

The protocol class for all staves. 

\Definitarg {:name}

The default value for this initarg is \lispobj{"default"}. 

\Defgeneric {name} {staff}

Return the name of the staff.  With \lispobj{setf}, change the name of
the staff.

\Defclass {fiveline-staff}

\Definitarg {:clef}

This value must always be supplied. 

\Definitarg {:keysig}

The default value for this initarg is a vector with seven elements,
each begin the object \lispobj{:natural}.

\Defun {make-fiveline-staff} {name \optional (clef \texttt(make-clef :treble))}

\Defgeneric {clef} {fiveline-staff}

Return the clef of the staff.  With \lispobj{setf}, change the clef of
the staff.

\Defgeneric {keysig} {fiveline-staff}

Return the key signature of the staff.  With \lispobj{setf}, change
the key signature of the staff. 

%-------------------------------------------------------------------
\subsection{External representation}

A fiveline staff is printed (by \lispobj{print-object}) like this in
version 3 of the external representation:

\texttt{[= :name \textit{name} :clef \textit{clef} :keysig \textit{keysig} ]}

The reader accepts this syntax, except that the slots can come in any
arbitrary order. 

In version 2 of the external representation, a staff was written like
this :

\texttt{[= \textit{clef} \textit{keysig} ]}

%===================================================================
\section{The keysig protocol}

There is no keysig protocol yet.  But I would like to create one.  A
keysig would be a read-only object.  

%===================================================================
