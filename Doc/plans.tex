\chapter{Plans for the future}

\section{Minor issues}

There are tons of minor issues to fix.  Here is a partial list.

\subsection{Notation issues}

\begin{itemize}
\item Elements now have an additional property that lets the user
  position them horizontally with an offset relative to their normal
  aligned position, so that notes with an interval of a second form
  different layers do not overlap physically.  But we also have to
  introduce a set of primitives to alter this offset, and perhaps some
  visual indications that show the offset.  Perhaps we should show a
  thin yellow line where the alignment is whenever there is a nonzero
  offset.  I am suggesting yellow, because it would not destroy the
  overall visual impression of the score. 
\item The cursor is currently drawn by the drawing routine for the
  layer, but that routine is not supposed to know where the cursor
  is.  We need to separate out cursor drawing from layer drawing. 
\end{itemize}

\subsection{Implementation details}

\begin{itemize}
\item Playing as MIDI should create a temporary file with a unique
  name in \texttt{/tmp} as opposed a file in the current directory
  with a name that can clash with others. 
\end{itemize}

\section{Major issues}

\subsection{Implementation details}

\begin{itemize}
\item Different information might need to be attached conceptually to
  barlines, in particular: fermata, repeat bars, etc.  It would be
  natural to physically attach this information to the bar, indicating
  properties of the barline at the end of the bar, but there is a
  conceptual problem with this, since some of this information is not
  unique to the layer, but sometimes global to all layers that are
  traversed by the resulting barline and sometimes even (repeat bars?)
  to the entire score.  One could imagine a separate structure for the
  segment, indicating positions of repeat bars and such.  It would
  have the additional advantage that if a barline is deleted in a
  layer which has a repeat bar further on, the visual effect would be
  nicer.  Operating on this new structure should be possible from any
  layer. 
\item The fermata symbol is sometimes centered on a beam (tremolo).
  We currently have no way of attaching any information to a beam,  
\end{itemize}

\section{Minor projects}

\subsection{Default staff line for rests}

When using several voices (usually two) on one staff line, one would
like for rests to be placed on (say) the upper staff line for the
upper voice and (say) the lower staff line for the lower voice rather
than, as now, having all rests inserted by default on the middle staff
line.  It would be nice if a layer had a default staff line for rests
so that changing hte layer automatically changes the default staff
line for rests.  

\subsection{Default stem direction for a layer}

When using several voices on a staff line, one somtimes wants the
stems of the upper voice to be up and the ones of the lower voice to
be down.  It would be nice for a layer to have a default stem
direction so that changing the voice would automatically change the
stem direction.  

Here is a possible specification of such a feature:

A new element is still created according to the stem direction of the
input state without taking into account the stem direction of the
layer.  When a layer is displayed, the following final stem direction
is computed:

\begin{itemize}
\item if the stem direction of the element is not \emph{auto} then
  that stem direction is used;
\item if the stem direction of the element is \emph{auto}, but the
  default stem direction of the layer is not \emph{auto}, then the
  default stem direction of the layer is used;
\item otherwise, the stem direction is computed from the position of
  the notes of the element. 
\end{itemize}

\subsection{Other minor projects}

\begin{itemize}
\item multi-buffer, multi-frame
\item display additional information around current element (beaming
  information, stem information (auto or not)), etc
\end{itemize}

\section{Major projects}

\subsection{Improved spacing algorithm}

There are two aspects of the spacing algorithm:

\begin{enumerate}
\item Determining how to divide a sequence of measures into pages and
  lines based on the amount of space each measure will require on a
  given line,
\item Rendering the page(s) actually visible by the user at any given
  time. 
\end{enumerate}

We can spend relatively much time on rendering since it is supposedly
only done for a small fraction of the entire score and relatively
little time on dividing sequences of measures into pages and lines.
Also, we can accept a rough estimate of the space that a measure needs
for the purpose of dividing into pages and lines, whereas the
rendering phase needs to be very precise so as to avoid overlapping
characters. 

For rendering, we can assume that we have a bunch of \emph{lines},
each one being a \emph{sequence of measures}.  For this phase, one
idea would be to compute an \emph{elasticity function} for each
measure indicating how willing it is to stretch or compress given that
a certain force is applied to it.  In fact it would indicate the width
as a function of force applied.  Such a function would be piecewise
linear with rational coordinates to avoid rounding errors.  The
elasticity function of a line would be the sum of the elasticity
functions of the measures of the line, again a piecewise linear
function.  The combined function would then be solved for the line
length which gives a force to be applied to each measure (elementary
mechanics give that each measure in the sequence would have the same
force applied to it).  To render each measure, that force would be
applied to it.  

The problem with dividing a sequence of measures into lines and pages
is that the elasticity function depends on the context.  Specifically,
the \emph{natural width} (the width that the measure would take when
given a force of 0) depends on the smallest distance between two
adjacent time lines \emph{of all of the time lines in all of the
measures on a line}.  This gets messy, since currently, we depend on
the cost function of a line to be combined in constant time, i.e., the
cost function of a line with a new measure added to the beginning or
to the end of it is possible to compute in constant time from the cost
function of the existing line and that of the new measure.  Either I
am being to conservative here, and we can afford to spend more time
computing the cost function of a line, or else, we need to find an
approximation of the elasticity function that will combine in constant
time.  This is hard to know until we have tested {\gs} on large
scores, and until we have implemented the page-breaking algorithm (as
opposed to the line-breaking algorithm that we now have). An example
of an approximation of the elasticity function would be to use the
current estimate with an additional value called \emph{additional
constant space}.  The cost combination function would add the constant
spaces required by each measure as long as newly added measures have
the same smallest time-line distance (min-dist).  As soon as a measure
with smaller min-dist is added to the line, the additional constant
space of the existing line is set to 0.  The assumption is that the
existing line is going to have to expand as a result of the smaller
min-dist which will eliminate the need for the additional constant
space.  Adding a measure with a larger min-dist than that of the
existing line would simply ignore the additional constant space of
that measure.  Again, the assumption is that the new measure will have
to expand so that the additional space will not be needed.

The additional constant spaced of a measure would be computed as if it
would be inserted into a line having the same min-dist as the measure
itself.  Such space would include space for lyrics, accidentals of
complicated clusters, etc. 

\subsection{Layout by page}

We need to have a more sophisticated layout algorithm that divides the
score into pages rather than just lines.  One interesting idea seems
to be to use the {\obseq} library in a nested kind of way, where an
invocation of the cost combination method on the global level would
provoke another call to the \texttt{solve} function on the page
level.  This would be really cute, actually, and it would show the
power of the approach of the {\obseq} library.  

Here is how it would be done.  First it requires a minor modification
to the {\obseq} library.  Instead of the elements \emph{inheriting}
from \texttt{obseq-elem}, they need to be separate so that an element
can be contained in more than one \texttt{obseq}.  This modification
has some consequences: the operations to compute the next and previous
element need to take an \texttt{obseq} argument to dispatch on.  The
operations \texttt{last-undamaged-element} and
\texttt{first-undamaged-element} must be modified and renamed to (say)
\texttt{tail-valid-elements} and \texttt{head-valid-elements} and take
a number instead of an element, since the \texttt{obseq} itself does
not have a mapping from its internal elements to the client element.
This requires the client to know how many elements there are, which is
a minor inconvenience (but still an inconvenience). Then the seq-cost
on the global level will be an obseq on the page level, and a call to
combine a seq-cost and an element on the global level would trigger a
call to solve on the page level.

We also need to have different cost combination strategies, for
instance:

\begin{itemize}
\item Paper roll with last line not right justified.  This involves
  assigning a low cost to the last line even when it has to be highly
  stretched in order to fill the line.  
\item Paper roll with last line right justified.  This is all we have
  right now.
\item Pages with last page not filled and last line on last page not
  filled. 
\item Pages with last page filled normally.
\item Pages with a fixed number of filled pages.
\item Pages with an arbitrary number of pages, but the number must be
  of the form $n = ax + b$.
\end{itemize}

\subsection{Other major projects}

\begin{itemize}
\item presentations everywhere (note heads, clusters, beam groups).
  Currently, McCLIM is a bit to slow for this.  
\item context menus on notes, etc
\item allow mouse-based input of new notes by making staff steps
  around cursor into presentations.  Move pointer to horizontal
  location of cursor after interaction.
\item allow mouse-based addition and deletion of notes in existing
  clusters by making staff steps around cluster into presentations
\item make sure user can use either keyboard or mouse without changing
  too often
\item add new views of score, especially to manipulate staves,
  brackets, braces, instruments, etc.
\item perhaps hide interactor by default and use something like M-x to
  make it visible. 
\item use output recording (hierarchical records) to reorganize pixmaps
  and to substitute combined pixmaps. 
\end{itemize}

\section{McCLIM issues and projects}

\begin{itemize}
\item add better system for Emacs-style commands and key
  bindings 
\item fix problem of input focus being retained (also a
  problem for Goatee)
\item make sure gestures work as announced
\item perhaps make it possible to use something like M-x to change
  input focus
\item make setf of frame-layout work
\item multiple top-level frames (as in Franz CLIM).
\item remove compilation warnings in McCLIM
\item remove dead code
\item remove reasons for remarks such as XXX:, FIXME, etc, or at least
  put a name in their place
\end{itemize}